%%%%%%%%%%%%%%%%%%%%%%%%%%%%%%%%%%%%%%%%%
% Short Sectioned Assignment
% LaTeX Template
% Version 1.0 (5/5/12)
%
% This template has been downloaded from:
% http://www.LaTeXTemplates.com
%
% Original author:
% Frits Wenneker (http://www.howtotex.com)
%
% License:
% CC BY-NC-SA 3.0 (http://creativecommons.org/licenses/by-nc-sa/3.0/)
%
%%%%%%%%%%%%%%%%%%%%%%%%%%%%%%%%%%%%%%%%%

%----------------------------------------------------------------------------------------
%	PACKAGES AND OTHER DOCUMENT CONFIGURATIONS
%----------------------------------------------------------------------------------------

\documentclass[paper=a4, fontsize=11pt]{scrartcl} % A4 paper and 11pt font size

%\usepackage[T1]{fontenc} % Use 8-bit encoding that has 256 glyphs
\usepackage[utf8]{inputenc}
\usepackage{fourier} % Use the Adobe Utopia font for the document - comment this line to return to the LaTeX default
\usepackage[portuguese]{babel} % English language/hyphenation
\usepackage{amsmath,amsfonts,amsthm} % Math packages

\usepackage{lipsum} % Used for inserting dummy 'Lorem ipsum' text into the template

\usepackage{sectsty} % Allows customizing section commands
\allsectionsfont{\centering \normalfont\scshape} % Make all sections centered, the default font and small caps

\usepackage{fancyhdr} % Custom headers and footers

\usepackage{url}
\usepackage{graphicx}
\usepackage{caption}
\usepackage{subcaption}

\pagestyle{fancyplain} % Makes all pages in the document conform to the custom headers and footers
\fancyhead{} % No page header - if you want one, create it in the same way as the footers below
\fancyfoot[L]{} % Empty left footer
\fancyfoot[C]{} % Empty center footer
\fancyfoot[R]{\thepage} % Page numbering for right footer
\renewcommand{\headrulewidth}{0pt} % Remove header underlines
\renewcommand{\footrulewidth}{0pt} % Remove footer underlines
\setlength{\headheight}{13.6pt} % Customize the height of the header

\numberwithin{equation}{section} % Number equations within sections (i.e. 1.1, 1.2, 2.1, 2.2 instead of 1, 2, 3, 4)
\numberwithin{figure}{section} % Number figures within sections (i.e. 1.1, 1.2, 2.1, 2.2 instead of 1, 2, 3, 4)
\numberwithin{table}{section} % Number tables within sections (i.e. 1.1, 1.2, 2.1, 2.2 instead of 1, 2, 3, 4)

\setlength\parindent{0pt} % Removes all indentation from paragraphs - comment this line for an assignment with lots of text

%----------------------------------------------------------------------------------------
%	TITLE SECTION
%----------------------------------------------------------------------------------------

\newcommand{\horrule}[1]{\rule{\linewidth}{#1}} % Create horizontal rule command with 1 argument of height

\title{	
\normalfont \normalsize 
\textsc{Universidade Federal do Paraná} \\ [25pt] % Your university, school and/or department name(s)
\horrule{0.5pt} \\[0.4cm] % Thin top horizontal rule
\huge Reconhecimento de Padrões \\
\large Trabalho final  % The assignment title
\horrule{2pt} \\[0.5cm] % Thick bottom horizontal rule
}

\author{Gian Maurício Fritsche} % Your name

\date{\normalsize\today} % Today's date or a custom date

\begin{document}

\maketitle % Print the title

%----------------------------------------------------------------------------------------
%	PROBLEM 1
%----------------------------------------------------------------------------------------

\section{Descrição do trabalho}

Para a realização deste trabalho foi utilizada a base de imagens SIMPSONS\footnote{\url{http://www.inf.ufpr.br/lesoliveira/padroes/simpsons.zip}}.
O objetivo é construir um sistema de reconhecimento de padrões que discrimine as cinco classes (representando os cinco personagens principais: Bart, Homer, Lisa, Maggie e Marge).

\begin{figure}
    \centering
    \begin{subfigure}[b]{0.15\textwidth}
        \includegraphics[width=\textwidth]{bart001}
        \caption{Bart}
        \label{fig:bart}
    \end{subfigure}
    ~ %add desired spacing between images, e. g. ~, \quad, \qquad, \hfill etc. 
      %(or a blank line to force the subfigure onto a new line)
    \begin{subfigure}[b]{0.15\textwidth}
        \includegraphics[width=\textwidth]{homer001}
        \caption{Homer}
        \label{fig:homer}
    \end{subfigure}
    ~ %add desired spacing between images, e. g. ~, \quad, \qquad, \hfill etc. 
    %(or a blank line to force the subfigure onto a new line)
    \begin{subfigure}[b]{0.15\textwidth}
        \includegraphics[width=\textwidth]{lisa001}
        \caption{Lisa}
        \label{fig:lisa}
    \end{subfigure}
    ~ %add desired spacing between images, e. g. ~, \quad, \qquad, \hfill etc. 
    %(or a blank line to force the subfigure onto a new line)
    \begin{subfigure}[b]{0.15\textwidth}
        \includegraphics[width=\textwidth]{maggie001}
        \caption{Maggie}
        \label{fig:maggie}
    \end{subfigure}
    ~ %add desired spacing between images, e. g. ~, \quad, \qquad, \hfill etc. 
    %(or a blank line to force the subfigure onto a new line)
    \begin{subfigure}[b]{0.15\textwidth}
        \includegraphics[width=\textwidth]{marge001}
        \caption{Marge}
        \label{fig:marge}
    \end{subfigure}
    \caption{Exemplos de imagens}\label{fig:exemplos}
\end{figure}

Na Figura~\ref{fig:exemplos} são apresentados exemplos de imagens para cada uma das cinco classes.
Foram utilizados três métodos para extração de características: Histograma de cor, Momentos de Hu e Histograma da orientação dos gradientes {\it Histogram of oriented gradients} (HOG).
Inicialmente toda imagem recebida pelo módulo de extração é redimensionada para $150 \times 150$, em seguida é enviada para o método de extração de características selecionado.

\subsection{Extração de características}

Para o histograma de cor, cada canal (cor de 0 à 255) foi dividida em quatro partes (bins) e calculado quantos pixels se encaixam em cada parte (para cada cor). 
Retornando assim um vetor de características com 64 posições.
A quantidade de divisões (bins) é um parâmetro do método, porém apenas o valor quatro foi avaliado.
Outro método de extração de características utilizado foi o Momentos de Hu, que são invariantes a translação, rotação e escala.
Para sua utilização a imagem foi convertida para tons de cinza. Este método retorna um vetor de sete características (os sete momentos de Hu).
É sugerido que este método seja utilizado após a segmentação da imagem, porém esta etapa não foi realizada.
O terceiro método de extração de características utilizado foi o HOG, que utiliza os gradientes da imagem para capturar contornos, silhuetas e algumas informações de textura.

\subsection{Classificação}

Para a classificação das imagens, inicialmente os vetores de característica são normalizados.
Em seguida é aplicado um dos três classificadores implementados: {\it Linear Discriminant Analysis} (LDA), {\it K-th Nearest Neighbor} (KNN) e {\it Support Vector Machine } (SVM).
O método KNN, apresenta dois parâmetros, o número de vizinhos ($k$) e a métrica de distância.
Os valores utilizados foram $k=5$ e distância euclidiana. Não foram avaliados outros valores.
Para o SVM o modelo foi aprendido por meio de {\it GridSearch}.

\subsection {Fusão}

Para a fusão foram construídas duas lista, uma com os métodos de extração de características ({\it features}) e outra com os classificadores ({\it clasifiers}).
Então para cada combinação $[feature, classifier]$ é construído e treinado um classificador.
Em seguida os exemplos de teste são classificados utilizando todos os classificadores construídos.
Para a fusão da saída dos classificadores foram implementados cinco métodos: soma, mínimo, máximo, produto e {\it borda count}.
Os quatro primeiros utilizam as probabilidades retornadas por cada classificador, 
enquanto para o {\it borda count} foi calculado os rankings (a partir das probabilidades).

\section {Análise da classificação}

A análise do modelo apresentado foi dividida em três etapas: Análise dos métodos de extração de características, análise dos métodos de classificação e análise dos métodos de fusão de classificadores. Para cada etapa são analisadas as matrizes de confusão e a taxa de acerto.

\subsection {Análise dos métodos de extração de características}

Para a análise de diferentes métodos de extração de características foi utilizado o classificador KNN.

\begin{table}[!htb]
\centering
\caption{Histograma de cor}
\label{tbl:colorhistogram}
\begin{tabular}{|l|l|l|l|l|l|l|}
\hline
\multicolumn{7}{|l|}{\textbf{Score 65.26\%}}                                          \\ \hline
            & bart      & homer      & lisa      & maggie     & marge    & score      \\ \hline
bart        & 32        & 2          & 1         & 0          & 0        & 91.43\%    \\ \hline
homer       & 7         & 12         & 5         & 1          & 0        & 48.00\%    \\ \hline
lisa        & 3         & 1          & 8         & 1          & 0        & 61.54\%    \\ \hline
maggie      & 4         & 1          & 0         & 7          & 0        & 58.33\%    \\ \hline
marge       & 7         & 0          & 0         & 0          & 3        & 30.00\%    \\ \hline
            &           &            &           &            & media    & 57.86\%    \\ \hline
\end{tabular}
\end{table}

\begin{table}[!htb]
\centering
\caption{Momentos de Hu}
\label{tbl:humoments}
\begin{tabular}{|l|l|l|l|l|l|l|}
\hline
\multicolumn{7}{|l|}{\textbf{Score 37.89\%}}                                          \\ \hline
            & bart      & homer      & lisa      & maggie     & marge    & score      \\ \hline
bart        & 20        & 6          & 6         & 1          & 2        & 57.14\%    \\ \hline
homer       & 10        & 8          & 3         & 3          & 1        & 32.00\%    \\ \hline
lisa        & 10        & 0          & 0         & 2          & 1        & 0.00\%     \\ \hline
maggie      & 4         & 2          & 1         & 3          & 2        & 25.00\%    \\ \hline
marge       & 2         & 1          & 1         & 1          & 5        & 50.00\%    \\ \hline
            &           &            &           &            & media    & 32.83\%    \\ \hline
\end{tabular}
\end{table}

\begin{table}[!htb]
\centering
\caption{Histogram of oriented gradients}
\label{tbl:hog}
\begin{tabular}{|l|l|l|l|l|l|l|}
\hline
\multicolumn{7}{|l|}{\textbf{Score 49.47\%}}                                          \\ \hline
            & bart      & homer      & lisa      & maggie     & marge    & score      \\ \hline
bart        & 32        & 3          & 0         & 0          & 0        & 91.43\%    \\ \hline
homer       & 11        & 14         & 0         & 0          & 0        & 56.00\%    \\ \hline
lisa        & 8         & 4          & 1         & 0          & 0        & 7.69\%     \\ \hline
maggie      & 9         & 3          & 0         & 0          & 0        & 0.00\%     \\ \hline
marge       & 7         & 3          & 0         & 0          & 0        & 0.00\%     \\ \hline
            &           &            &           &            & media    & 31.02\%    \\ \hline
\end{tabular}
\end{table}

\subsection {Análise dos métodos de classificação}

Para a análise de diferentes métodos de classificação foi utilizado o método de extração de características Histograma de cor.
Dado que na análise dos métodos de extração de características este foi o que apresentou a maior taxa de acerto geral (65.26\%), e a maior taxa média de acerto entre as diferentes classes (57.86\%).

\begin{table}[!htb]
\centering
\caption{SVM}
\label{tbl:svm}
\begin{tabular}{|l|l|l|l|l|l|l|}
\hline
\multicolumn{7}{|l|}{\textbf{Score 62.11\%}}                   \\ \hline
         & bart   & homer   & lisa   & maggie   & marge  & score    \\ \hline
bart     & 27     & 7       & 1      & 0        & 0      & 77.14\%  \\ \hline
homer    & 5      & 15      & 3      & 1        & 1      & 60.00\%  \\ \hline
lisa     & 4      & 3       & 6      & 0        & 0      & 46.15\%  \\ \hline
maggie   & 1      & 3       & 0      & 8        & 0      & 66.67\%  \\ \hline
marge    & 5      & 1       & 1      & 0        & 3      & 30.00\%  \\ \hline
         &        &         &        &          & media  & 55.99\%  \\ \hline
\end{tabular}
\end{table}

\begin{table}[!htb]
\centering
\caption{LDA}
\label{tbl:lda}
\begin{tabular}{|l|l|l|l|l|l|l|}
\hline
\multicolumn{7}{|l|}{\textbf{Score 68.42\%}}                   \\ \hline
         & bart   & homer   & lisa   & maggie   & marge  & score    \\ \hline
bart     & 30     & 4       & 1      & 0        & 0      & 85.71\%  \\ \hline
homer    & 2      & 16      & 6      & 0        & 1      & 64.00\%  \\ \hline
lisa     & 2      & 3       & 8      & 0        & 0      & 61.54\%  \\ \hline
maggie   & 2      & 2       & 0      & 8        & 0      & 66.67\%  \\ \hline
marge    & 3      & 1       & 3      & 0        & 3      & 30.00\%  \\ \hline
         &        &         &        &          & media  & 61.58\%  \\ \hline
\end{tabular}
\end{table}

\begin{table}[!htb]
\centering
\caption{KNN}
\label{tbl:knn}
\begin{tabular}{|l|l|l|l|l|l|l|}
\hline
\multicolumn{7}{|l|}{\textbf{Score 65.26\%}}                   \\ \hline
         & bart   & homer   & lisa   & maggie   & marge  & score    \\ \hline
bart     & 32     & 2       & 1      & 0        & 0      & 91.43\%  \\ \hline
homer    & 7      & 12      & 5      & 1        & 0      & 48.00\%  \\ \hline
lisa     & 3      & 1       & 8      & 1        & 0      & 61.54\%  \\ \hline
maggie   & 4      & 1       & 0      & 7        & 0      & 58.33\%  \\ \hline
marge    & 7      & 0       & 0      & 0        & 3      & 30.00\%  \\ \hline
         &        &         &        &          & media  & 57.86\%  \\ \hline
\end{tabular}
\end{table}

\end{document}